\documentclass[ngerman, 11pt, a4paper, twoside, abstracton]{scrbook}

\usepackage[utf8]{inputenc}
\usepackage[T1]{fontenc}
\usepackage[english, ngerman, main=ngerman]{babel}

\usepackage{hyperref}

\usepackage{csquotes}
\usepackage{amsmath}
\usepackage{amssymb}
%\usepackage{amsthm}
\usepackage[amsmath, amsthm, thmmarks]{ntheorem}
\usepackage{cleveref}
\usepackage{mathtools}
\usepackage{enumerate}
\usepackage{wasysym}

\parskip=6pt
\parindent=0pt

%\pagestyle{headings}

\newcommand{\Q}{\mathbb{Q}}
\newcommand{\R}{\mathbb{R}}
\newcommand{\C}{\mathbb{C}}

\theoremstyle{break}
\theorembodyfont{\normalfont}
\theoremsymbol{\ensuremath{\Diamond}}
\theoremseparator{.}
\newtheorem{definition}{Definition}

\theoremsymbol{\ensuremath{\smiley}}
\newtheorem{pattern}{Idee}

\theoremsymbol{\ensuremath{\Diamond}}
\newtheorem{exercise}{Aufgabe}




\begin{document}

\title{Mathematik}
\author{Paul Reichert}
\date{\today}

\maketitle

\chapter{Algebra}

\section{Galoistheorie, Ordnung und Ganzheit}

\subsection{Überblick und Leitmotive}

Nicht wegzudenken aus der Algebra sind Polynome. Wenn es also darum geht, Ringe und Körper zu untersuchen, dann wäre es schön, wenn wir das mit Polynomen machen können. Wir interessieren uns insbesondere für sogenannte Erweiterungen von Ringen und Körpern, die es uns erlauben, neue Elemente zu unserem algebraischen Objekt hinzuzunehmen. Bekannte Beispiele dafür sind die reellen Zahlen als Körpererweiterung der rationalen Zahlen $\R\supseteq\Q$ oder -- durch Hinzunahme der imaginären Einheit -- $\C\supseteq\R$.

Doch was ist eigentlich mit dem Begriff \enquote{Erweiterung} gemeint? Um das Verständnis zu schärfen, eine Definition:

\begin{definition}[Erweiterung]
  \begin{enumerate}[(a)]
    \item Sei $R$ ein Ring. Eine Ringerweiterung von $R$ ist ein Ring $S$, der $R$ als Teilring enthält.
    \item Eine Ringerweiterung heißt Körpererweiterung, wenn die beteiligten Ringe Körper sind.
  \end{enumerate}

  In beiden Fällen schreibt man die Erweiterung als $Y\supseteq X$.
\end{definition}

Man kann auf eine solche Erweiterung auf verschiedene Weisen schauen. Zunächst manifestiert sich in $Y\supseteq X$, dass $Y$ eine $X$"=Algebra ist. Oft interessieren wir uns aber für \emph{als $X$"=Modul} endlich erzeugte Erweiterungen.

Der Grund liegt in der folgenden Beobachtung:

\begin{pattern}[algebraische Elemente]
  Sei $S\supseteq R$ eine Ringerweiterung und $s\in S$.

  \begin{enumerate}[(a)]
    \item $R\left[ s \right]$ ist genau dann als $R$"=Modul endlich erzeugt, wenn $s$ Nullstelle eines normierten Polynoms $f\in R\left[ X \right]$ ist.
    \item Seien nun $R$ und $S$ Körper. Dann ist $R\left( s \right)$ genau dann ein endlichdimensionaler $R$"=Vektorraum, wenn $s$ Nullstelle eines Polynoms $f\in R\left[ X \right]\setminus\left\{ 0 \right\}$ ist.
  \end{enumerate}

\end{pattern}

Der Grund liegt rein intuitiv in den folgenden zwei Beobachtungen. Ist $s$ Nullstelle eines passenden Polynoms, kann $R\left[ s \right]$ als Faktoralgebra von $R\left[ X \right]$ geschrieben werden. Ist andererseits $R\left[ s \right]$ ein endlichdimensionaler $R$"=Modul, sind $1, s, s^2,\ \dots$ linear unabhängig. Hat man eine lineare Abhängigkeit gefunden, dann hat man ein passendes annullierendes Polynom gefunden. Das ist kein Beweis, aber eine gute Vorbereitung auf das Kommende.

\chapter{Funktionalanalysis}

\section{Ein Pattern für Banachraum-Beweise}

%\newcommand{\norm}[1]{\lVert #1 \rVert}
\newcommand{\inorm}[1]{\norm{#1}_\infty}

Der Banachraum $\left(C\left[ 0, 1 \right], \inorm{\cdot}\right)$ lässt sich auf viele interessante und nützliche Arten verfeinern.

\begin{itemize}
  \item Für $\alpha\in\left( 0, 1 \right)$ sei
    \[
      \norm{f}_{C^\alpha} \coloneqq \inorm{f} + \sup_{\substack{x, y\in \left[ 0, 1 \right] \\ x \ne y}} \frac{\abs{f\left( x \right) - f\left( y \right)}}{\abs{x-y}^\alpha}.
    \]
    Die Funktionen $f\in C\left[ 0, 1 \right]$ mit $\norm{f}_{C^\alpha}<\infty$ bilden den normierten Raum $C^\alpha\left[ 0, 1 \right]$ der $\alpha$-Hölder-stetigen Funktionen.
  \item Es sei
    \[
      \norm{f}_{BV} \coloneqq \inorm{f} + \sup_{0\le x_0 \le \ldots \le x_L \le 1} \sum_{l=1}^L \abs{f\left( x_l \right) - f\left( x_{l-1} \right)}.
    \]
    Die Funktionen $f\in C\left[ 0, 1 \right]$ mit $\norm{f}_{BV}<\infty$ bilden den normierten Raum der stetigen Funktionen beschränkter Variation.
\end{itemize}

In beiden Beispielen wird aus $\inorm{\cdot}$ eine neue Norm konstruiert, indem eine Seminorm dazuaddiert wird, die den Konvergenzbegriff, der von der Norm induziert wird, verschärft.

Oft erhält man dadurch einen Banachraum. Auf die Beweisidee zu kommen braucht etwas, aber hier ist ein Beispiel für einen solchen Beweis.

\begin{lemma}
  $\left( C^\alpha\left[ 0, 1 \right], \norm{\cdot}_{C^\alpha} \right)$ ist ein Banachraum.
\end{lemma}
\begin{proof}
  Definiere die Seminorm
  \[
    q_\alpha \coloneqq \norm{f}_{C^\alpha} - \norm{f}_\infty = \sup_{\substack{x,y\in\left[ 0, 1 \right] \\ x\ne y}} \frac{\abs{f\left( x \right) - f\left( y \right)}}{\abs{x-y}^\alpha}.
  \]

  Sei $\left( f_n \right)$ eine Cauchyfolge in $C^\alpha\left[ 0, 1 \right]$. Dann ist $\left( f_n \right)$ auch eine Cauchyfolge in $C\left[ 0, 1 \right]$. Also existiert ein $f\in C\left[ 0, 1 \right]$ mit $\inorm{f-f_n}\to 0$. Es bleibt zu zeigen: $q_\alpha\left( f-f_n \right) \to 0$.

  Wir werden gleich zeigen, dass für eine beliebige gleichmäßig konvergente Folge $\left( g_n \right)$ mit $g_n\in C^\alpha \left[ 0, 1 \right]$ und $g_n \to g$ folgende Ungleichung gilt:
  \begin{align*}
    q_\alpha\left( g \right) \le \liminf q_\alpha\left( g_n \right). && \tag{$\ast$}.
  \end{align*}
  Daraus folgt dann für $m \ge n$
  \[
    q_\alpha\left( f - f_n \right) \le \liminf_m q_\alpha\left( f_m - f_n \right) \le \liminf_m \norm{f_m - f_n}_{C^\alpha} \le C_n,
  \]
  wobei $C_n$ eine Nullfolge ist ($\left( f_n \right)$ ist Cauchyfolge).

  Dies liefert $\norm{f - f_n}_{C^\alpha} \to 0$. Damit liegt $f$ sogar in $C^\alpha\left[ 0, 1 \right]$. Darum ist $C^\alpha\left[ 0, 1 \right]$ ein Banachraum.

  Nun zum Beweis von $\left( \ast \right)$.

  Für feste, unterschiedliche $x, y \in \left[ 0, 1 \right]$ und $n\in\mathbb{N}$ gilt mit der Dreiecksungleichung
  \[
    \frac{\abs{f\left( x \right) - f\left( y \right)}}{\abs{x-y}^\alpha} \le \frac{\abs{f_n\left( x \right) - f_n\left( y \right)} + 2 \inorm{f - f_n}}{\abs{x-y}^\alpha}.
  \]

  Übergang zum Limes inferior über $n\in N$ liefert 
  \[
    \frac{\abs{f\left( x \right)-f\left( y \right)}}{\abs{x-y}^\alpha} \le q_\alpha\left( f_n \right).
  \]

  Die Behauptung $\left( \ast \right)$ folgt dann durch Übergang zum Supremum über $x$ und $y$.
\end{proof}

\section{Satz von Baire}

\begin{theorem}[Baire, etwas umformuliert]
  Sei $\left( X, d \right)$ ein vollständiger metrischer Raum und $\left( A_n \right)$ eine Folge abgeschlossener Mengen in $X$. Enthält $\bigcup A_n$ einen inneren Punkt, dann enthält schon eines der $A_n$ einen inneren Punkt.
\end{theorem}
\begin{proof}
  Annahme: Kein $A_n$ enthält einen inneren Punkt.

  Sei $x$ ein innerer Punkt von $A\coloneqq\bigcup A_n$ und $U_0$ eine offene Umgebung um $x$, die in $A$ enthalten ist. Wie in jedem metrischen Raum existieren $D_0$ abgeschlossen und $U_1$ nichtleer und offen mit $U_1 \subseteq D_0 \subseteq U_0$.
  
  Da $A_0$ keinen inneren Punkt enthält, ist die offene Menge $U_1 \setminus A_0$ nichtleer: $U_1$ kann damit auch disjunkt zu $A_1$ gewählt werden, wovon wir nun ausgehen.

  Induktiv können wir so Folgen $\left( D_n \right), \left( U_n \right)$ nichtleerer abgeschlossener respektive offener Mengen mit
  \[
    U_0 \supseteq D_0 \supseteq U_1 \supseteq D_1 \supseteq \cdots
  \]
  konstruieren, wobei $U_n$ und $A_n$ stets disjunkt sind.

  Dann gilt $\bigcap U_n = \bigcap D_n$. Der Cantorsche Durchschnittssatz besagt, dass $\bigcap D_n \ne \emptyset$ gilt; andererseits folgt aus $U_n \cap A_n = \emptyset$, dass $\left( \bigcap U_n \right) \cap X = \left( \bigcap U_n \right) \cap \left( \bigcup A_n \right) = \emptyset$.

  Das ist ein Widerspruch.
\end{proof}
\appendix

\chapter{Interessante Aufgaben}

\section{Algebra}

\begin{exercise}
  Sei $R$ ein noetherscher Ring, der Krull"=Dimension 0 hat, dessen minimale Primideale also gerade die maximalen sind. Dann hat $R$ nur endlich viele Primideale.
\end{exercise}
\begin{proof}
  Sei $\mathfrak{p}$ irgendein Primideal in $R$. Wir konstruieren induktiv eine aufsteigende Kette, die in $\mathfrak{p}$ stationär wird.

  Als kleine Vorbereitung wählen wir zu jedem Ideal $I$, das kein Primideal ist, zwei feste $s_I, t_I\in R\setminus I$, sodass $s_It_I\in I$.

  Setze $I_0 \coloneqq \left( 0 \right)$. Entweder ist $\left( 0 \right) = \mathfrak{p}$ oder $\left( 0 \right)$ kann nach Voraussetzung kein Primideal sein. Da $\mathfrak{p}$ ein Primideal über $I_0$ ist, gibt es also ein Element $r\in\{s_{I_0}, r_{I_0} \}$, das in $\mathfrak{p}$ liegt, und wir definieren $I_1\coloneqq \left( I_0, r \right)$. Solange kein $I_n$ gleich $\mathfrak{p}$ ist, ließe sich so eine unendlich strikt aufsteigende Kette bilden. Weil $R$ nach Voraussetzung aber noethersch ist, wird nach endlich vielen Schritten $\mathfrak{p}$ erreicht.

  Was man daran erkennt, ist, dass die Primideale die Blätter eines noetherschen Binärbaums ist, dessen Wurzel $\left( 0 \right)$ ist, also eines Binärbaums, in dem jeder Pfad nur endliche Länge hat.

  Angenommen, ein noetherscher Binärbaum hätte unendliche Tiefe. Dann ließe sich darin ein unendlicher Pfad konstruieren, indem immer solche Kanten gewählt werden, dass der Teilbaum unter dieser Kante auch noch unendliche Tiefe hätte. Das ist ein Widerspruch!

  Sei nun $t\in \mathbb{N}_0$ die endliche Tiefe unseres Binärbaums. Dann gibt es maximal $2^t$ Primideale (= Blätter) in $R$.
\end{proof}

\section{Topologie}

\begin{exercise}
  Sei $S^\infty$ die Vereinigung aller Sphären (nicht disjunkt), versehen mit der schwachen Topologie. Zeige: $S^\infty$ ist kontrahierbar.
\end{exercise}

\begin{exercise}
  Sei $n\in\mathbb{N}$. Dann ist $\mathbb{P}^n$ hausdorffsch.
\end{exercise}
\begin{proof}
  Seien $x, y \in \mathbb{P}^n$ verschieden. Wir unterscheiden zwei nicht notwendigerweise disjunkte Fälle und finden disjunkte offene Umgebungen.

  \paragraph{Fall 1} In homogenen Koordinaten gibt es ein $i$ mit $x_i \ne 0 \ne y_i$.
  
  $U_i \coloneqq \left\{ x \in \mathbb{P}^n \mid x_i \ne 0 \right\}$ ist offen und homöomorph zum $\mathbb{R}^n$, also hausdorffsch. Daher gibt es disjunkte offene Umgebungen um $x$ und um $y$.
  \paragraph{Fall 2} Es gibt $i$ und $j$ mit $x\in U_i \setminus U_j$ und $y\in U_j \setminus U_i$.

  Definiere $U_{ij} \coloneqq \left\{ z\in U_i \cap U_j \mid \abs{\frac{z_i}{z_j}} > 1\right\}$ (man sollte sich vergewissern, dass $U_{ij}$ auf den homogenen Koordinaten wohldefiniert ist!).

  Seien $\left[ x_0, \ldots, x_n \right]$ die homogenen Koordinaten von $x$, ohne Einschränkung mit $x_i = 1$. Aus $x_j = 0$ folgt, dass
  \[
    V_x \coloneqq \left\{ \left[ z_0, \ldots, z_n \right] \mid z_i = 1 \land z_j \in \left( -\frac{1}{2}, \frac{1}{2} \right) \right\}
  \]
  eine offene Umgebung von $x$ in $U_i$ ist, wobei $V_x \cap U_j \subseteq U_{ij}$ gilt.

  Analog lassen sich $U_{ji}$ und $V_y$ definieren. Dann
  \[
    V_x \cap V_y = \left( V_x \cap U_i \right) \cap \left( V_y \cap U_j \right) \subseteq U_{ij} \cap U_{ji} = \emptyset.
  \]

  Die Umgebungen $V_x$ und $V_y$ leisten also das Gewünschte.
\end{proof}
\section{Algebraische Geometrie}

\begin{exercise}
  Consider a scheme $X$ with structure sheaf $\mathcal O_X$. Suppose that for each affine open $U \cong \Spec \mathcal O_X\left( U \right)$, $I\left( U \right) \subseteq \mathcal O_X\left( U \right)$ is an ideal and that these ideals are compatible with localization, i.\ e.\ $I\left( U \right)_f \cong I\left( U_f \right)$ via the canonical isomorphism $\Spec \left(\mathcal O_X\left( U \right)_f\right) \cong U_f$.

  Show that the affine closed embeddings $U/I \coloneqq \Spec \mathcal O_X\left( U \right)/I\left( U \right) \hookrightarrow \Spec\mathcal O_X\left( U \right)$ glue to give rise to a closed embedding $Y\hookrightarrow X$.
\end{exercise}
\begin{proof}
  Let us show that the affine closed embeddings can be glued in a canonical way. Therefore note that given two affine open subsets $U_1, U_2$ of $X$, we have canonical isomorphisms of schemes
  \[
    U_i \overset{\varphi\left( U_i \right)}{\cong} \Spec\left( \mathcal O_X\left( U_i \right) \right).
  \]

  It is known to me that $U_1\cap U_2$ is covered by the open sets which are distinguished in both $U_i$ simultaneously.

  As a first step to gluing note that purely topologically, the $U/I$ cover the closed subset $V\left( I \right)$. We need to construct isomorphisms \textit{of schemes} on the overlaps such as $U_1 \cap U_2$.

  Now consider a simultaneously distinguished set $D \subseteq U_1\cap U_2$. Then $D$ is affine open and $D/I$ is well"=defined. The schemes of the form $D/I$ cover $U_1 \cap U_2$ topologically. 

  The isomorphism $\varphi\left( U_i \right)\left( D \right) \cong D \cong \varphi\left( U_j \right)\left( D \right)$ induces
  \[
    (U_i/I)_f \cong D/I \cong (U_j/I)_g.
  \]

  and for two simultaneously distinguished sets $D_1, D_2$, these isomorphisms agree on their intersection. Hence they can be glued to an isomorphism of $U_1/I$ and $U_2/I$ on their intersection.

  It remains to show that the cocycle condition is satisfied, which means that the isomorphisms agree on triple intersections. But this follows from the fact that they are induced by the $\varphi\left( U_i \right)$ which in turn do agree on triple intersections.
\end{proof}

\section{Funktionalanalysis}

\end{document}
