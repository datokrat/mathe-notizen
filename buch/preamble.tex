\usepackage[utf8]{inputenc}
\usepackage[T1]{fontenc}
\usepackage[english, ngerman, main=ngerman]{babel}

\usepackage{hyperref}

\usepackage{csquotes}
\usepackage{amsmath}
\usepackage{amssymb}
%\usepackage{amsthm}
\usepackage[amsmath, amsthm, thmmarks]{ntheorem}
\usepackage{cleveref}
\usepackage{mathtools}
\usepackage{enumerate}
\usepackage{wasysym}
\usepackage{physics}

\parskip=6pt
\parindent=0pt

%\pagestyle{headings}

\newcommand{\Q}{\mathbb{Q}}
\newcommand{\R}{\mathbb{R}}
\newcommand{\C}{\mathbb{C}}
\DeclareMathOperator{\Spec}{Spec}

\theoremstyle{break}
\theorembodyfont{\normalfont}
\theoremsymbol{\ensuremath{\Diamond}}
\theoremseparator{.}
\newtheorem{definition}{Definition}

\theoremsymbol{\ensuremath{\Box}}
\newtheorem{lemma}{Lemma}

\theoremsymbol{\ensuremath{\smiley}}
\newtheorem{pattern}{Idee}

\theoremsymbol{\ensuremath{\Diamond}}
\newtheorem{exercise}{Aufgabe}

